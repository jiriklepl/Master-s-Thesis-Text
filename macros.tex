
% use this for typesetting a chapter without a number, e.g. intro and outro
\def\chapwithtoc#1{
\chapter*{#1}
\addcontentsline{toc}{chapter}{#1}
}

% If there is a line/figure overflowing into page margin, this will make the
% problem evident by drawing a thick black line at the overflowing spot. You
% should not disable this.
\overfullrule=3mm

% The maximum stretching of a space. Increasing this makes the text a bit more
% sloppy, but may prevent the overflows by moving words to next line.
\emergencystretch=1em

\ifEN
\theoremstyle{plain}
\newtheorem{thm}{Theorem}
\newtheorem{lemma}[thm]{Lemma}
\newtheorem{claim}[thm]{Claim}
\newtheorem{defn}{Definition}
\newtheorem{observe}{Corollary}
\newtheorem{remark}{Remark}
\newtheorem{conv}{Convention}
\newtheorem{ex}{Example}
\theoremstyle{remark}
\newtheorem*{cor}{Corollary}
\else
\theoremstyle{plain}
\newtheorem{thm}{Věta}
\newtheorem{lemma}{Lemma}
\newtheorem{claim}{Tvrzení}
\newtheorem{defn}{Definice}
\newtheorem{observe}{Pozorování}
\newtheorem{remark}{Poznámka}
\newtheorem{conv}{Konvence}
\newtheorem{ex}{Ukázka}
\theoremstyle{remark}
\newtheorem*{cor}{Důsledek}
\fi

\newenvironment{myproof}{
  \par\medskip\noindent
  \textit{\ifEN Proof \else Důkaz \fi}.
}{
\newline
\rightline{$\qedsymbol$}
}

% real/natural numbers
\newcommand{\R}{\mathbb{R}}
\newcommand{\N}{\mathbb{N}}

% asymptotic complexity
\newcommand{\asy}[1]{\mathcal{O}(#1)}

% listings and default lstlisting config (remove if unused)
\DeclareNewFloatType{listing}{}
\floatsetup[listing]{style=ruled}
\DeclareNewFloatType{codex}{}
\floatsetup[codex]{style=ruled}
\DeclareNewFloatType{lang}{}
\floatsetup[lang]{style=Boxed}

\counterwithin{codex}{chapter}
\counterwithin{listing}{chapter}
\counterwithin{algorithm}{chapter}
\counterwithout{figure}{chapter}


\DeclareCaptionStyle{thesis}{style=base,font={small,sf},labelfont=bf,labelsep=quad}
\DeclareCaptionStyle{thesisbox}{style=boxed,font={small,sf},labelfont=bf,labelsep=quad}
\captionsetup{style=thesis}
\captionsetup[algorithm]{style=thesis,singlelinecheck=off}
\captionsetup[listing]{style=thesis,singlelinecheck=off}
\captionsetup[codex]{style=thesis,singlelinecheck=off}
\captionsetup[lang]{style=thesis,singlelinecheck=off}

% Uncomment for table captions on top. This is sometimes recommended by the
% style guide, and even required for some publication types.
%\floatsetup[table]{capposition=top}
%
% (Opinionated rant:) Captions on top are not "compatible" with the general
% guideline that the tables should be formatted to be quickly visually
% comprehensible and *beautiful* in general (like figures), and that the table
% "head" row (with column names) should alone communicate most of the content
% and interpretation of the table. If you just need to show a long boring list
% of numbers (because you have to), either put some effort into showing the
% data in an attractive figure-table, or move the data to an attachment and
% refer to it, so that the boredom does not impact the main text flow.
%
% You can make the top-captions look much less ugly by aligning the widths of
% the caption and the table, with setting `framefit=yes`, as shown below.  This
% additionally requires some extra markup in your {table} environments; see the
% comments in the example table in `ch2.tex` for details.
%\floatsetup[table]{capposition=top,framefit=yes}

\ifEN\floatname{listing}{Listing}
\else\floatname{listing}{Výpis kódu}\fi
\ifEN\floatname{codex}{Code sample}
\else\floatname{codex}{Ukázka kódu}\fi
\ifEN\floatname{lang}{Language}
\else\floatname{lang}{Jazyk}\fi

\lstset{ % use this to define styling for any other language
  language=C,
  tabsize=2,
  showstringspaces=false,
  morekeywords={continuation,cut,to,foreign,dropped,bits32,bits64, class,instance,ptr,new,stackdata},
  deletekeywords={while},
  basicstyle=\footnotesize\tt\color{black!75},
  identifierstyle=\bfseries\color{black},
  commentstyle=\color{green!50!black},
  stringstyle=\color{red!50!black},
  keywordstyle=\color{blue!75!black}
}

\lstdefinestyle{haskellStyle}{
  language=Haskell,
  morekeywords={Foldable,Traversable,Data,forall}
}

\lstdefinestyle{llvmStyle}{
  language=LLVM,
  morekeywords={undef, ccc}
}

% Czech versions of the used cleveref references (It's not as convenient as in
% English because of declension, cleveref is limited to sg/pl nominative. Use
% plain \ref to dodge that.)
\ifEN
%\crefname{defn}{definition}{definitions}
\Crefname{defn}{Definition}{Definitions}
%\crefname{lang}{language}{languages}
\Crefname{lang}{Language}{Languages}
%\crefname{codex}{code sample}{code samples}
\Crefname{codex}{Code sample}{Code samples}
\else
\crefname{chapter}{kapitola}{kapitoly}
\Crefname{chapter}{Kapitola}{Kapitoly}
\crefname{section}{sekce}{sekce}
\Crefname{section}{Sekce}{Sekce}
\crefname{subsection}{sekce}{sekce}
\Crefname{subsection}{Sekce}{Sekce}
\crefname{subsubsection}{sekce}{sekce}
\Crefname{subsubsection}{Sekce}{Sekce}
\crefname{figure}{obrázek}{obrázky}
\Crefname{figure}{Obrázek}{Obrázky}
\crefname{table}{tabulka}{tabulky}
\Crefname{table}{Tabulka}{Tabulky}
\crefname{listing}{výpis}{výpisy}
\Crefname{listing}{Výpis}{Výpisy}
\floatname{algorithm}{Algoritmus}
\crefname{algorithm}{algoritmus}{algoritmy}
\Crefname{algorithm}{Algoritmus}{Algoritmy}
\newcommand{\crefpairconjunction}{ a~}
\newcommand{\crefrangeconjunction}{ a~}
\fi


\providecommand{\free}[1]{\mtt{free} (#1)}
\providecommand{\dom}[1]{\mtt{Dom} (#1)}
\providecommand{\ran}[1]{\mtt{Ran} (#1)}
\def\cmm{\mbox{C-\phantom{}-}\xspace}
\def\cmmrepo{\footnote{The prototype compiler and its documentation are available at: \url{https://github.com/jiriklepl/masters-thesis-code}; more information given in \cref{chap:proto}}\xspace}
\providecommand{\vect}[1]{\overrightarrow{#1}}
\renewcommand{\algorithmiccomment}[1]{\hfill\textcolor{green!50!black}{$\triangleright$~\textit{#1}}}

% grammar tuning
\def\grammarP{0.666}
\def\firstcasesubtilP#1#2#3{\firstcasesubtil{#1}{#2}{\parbox[t]{\grammarP\linewidth}{#3}}}
\def\firstcaseP#1#2#3{\firstcase{#1}{#2}{\parbox[t]{\grammarP\linewidth}{#3}}}
\def\otherformP#1#2{\otherform{#1}{\parbox[t]{\grammarP\linewidth}{#2}}}
