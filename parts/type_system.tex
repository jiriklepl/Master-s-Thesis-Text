\documentclass[12pt,a4paper]{article}
\usepackage[a-2u]{pdfx}

%% Character encoding: usually latin2, cp1250 or utf8:
\usepackage[utf8]{inputenc}

%% Prefer Latin Modern fonts
\usepackage{lmodern}

\usepackage{amsmath}        % extensions for typesetting of math
\usepackage{amsfonts}       % math fonts
\usepackage{mathtools}       
\usepackage{array}       
\usepackage{amsthm}         % theorems, definitions, etc.
% \usepackage{bbding}         % various symbols (squares, asterisks, scissors, ...)
\usepackage{bm}             % boldface symbols (\bm)
\usepackage{algpseudocode}
\usepackage{bussproofs}

\usepackage{float}

\newtheorem{definition}{Definition}
\newtheorem{observation}{Observation}
\newtheorem{lemma}{Lemma}

\newenvironment{grammar}[2]
 {\begin{tabular}{@{\qquad}>{$}l<{$}@{\qquad}p{.75\linewidth}@{}}
  \multicolumn{1}{@{}l@{}}{$#1$}&\multicolumn{1}{l@{}}{\hspace{-2em}#2}\\}
 {\end{tabular}}

\begin{document}

\section{Type System}

The type system is based on an extension of the ML type system with (multi-parameter) type classes and type constructors with monomorphic type kinds.

This is extended by adding a limited form of subtyping. In the new type system, the types are a result of the traditional typing together with added subtype dimensions and their subtype constraints.

\begin{definition}[Typing dimension]
    The typing dimension is a semilattice with type unification serving as the meet operation.
\end{definition}

\begin{definition}[Subtype dimension]
    The subtype dimension $\mathbb{S}$ is required to be a bounded lattice independent on the type-inferred program. We then define a transitive and reflective binary relation on types $\leq_S$ ($\tau \leq_S \sigma \iff \tau_S \wedge \sigma_S = \tau_S\ \mathrm{and}\ \tau_S \vee \sigma_S = \sigma_S$, where $\tau_S$ and $\sigma_S$ are the types' respective subtype dimensions) which states that one type is less general in the given dimension than another type.

    We extend this definition to other comparison operators as well. Note that $\tau =_S \sigma \not\Rightarrow \tau = \sigma$, but indeed $\tau = \sigma \Rightarrow \tau =_S \sigma$.

    We use the operator $\leq_S$ as a type constraint and we allow one of the operands be an element of $\mathbb{S}$.

    In the type system specific to this thesis, we use two subtype dimensions: data kinds $\mathbb{K}$ and constnesses $\mathbb{C}$.
\end{definition}

\begin{observation}[Subtype dimension constraint combining]
    If a type $\tau$ is upper-bounded in a subtype dimension by multiple upperbounds: $\tau \leq_S s_1$, $\tau \leq_S s_2$, then we can replace such constraints with a single $\tau \leq_S (s_1 \wedge s_2)$. Similarly for lowerbounds and the join operation.

    Note that given multiple upperbounds and lowerbounds, the set of viable types can be empty. 
\end{observation}


\begin{definition}[Types]
    Let $\mathbb{S}_1, \dots \mathbb{S}_n$ be all subtype dimensions recognized by the system. Then for two types $\tau, \sigma$ with the same typing: \linebreak $\tau = \sigma \iff \forall i \in \{1, \dots n\} . \tau =_{S_n} \sigma$. 

    We then formally define types as a composition of their dimensions:
    $\tau = \tau_T + \tau_{S_1} + \cdots \tau_{S_n}$, where $\tau_T$ is the typing of $\tau$.
\end{definition}

\begin{definition}[Type variables]
    $\mathbb{V}$ is the space of type variables given as follows, each (sub)type considered by the inference algorithm will be assigned a type variable representing it:
    \begin{grammar}{\mathbb{V} \Rightarrow v}{}
        \mid \mathbb{V} '
    \end{grammar}

    The variables follow the obvious lexical ordering (based on their length), we use this ordering when choosing \textit{fresh variables}.
\end{definition}

\begin{definition}[Typings]
    $\mathbb{T}$ is the space of typings given as follows, they specify that a type follows the same typing as a different type:
    \begin{table}[H]
        \begin{grammar}{\mathbb{T} \Rightarrow }{Typing}
            \mathbb{V} & Variable \\
            \left[\mathbb{T}\right] \to \mathbb{T} & Function \\
            \mid \left[\mathbb{T}\right] & Tuple \\
            \mid \mathbb{T}\ \mathbb{T} & Application \\
            \cdots
        \end{grammar}
    \end{table}
\end{definition}

\begin{definition}[Types]
    Types use the same language as typings and thus the space $\mathbb{T}$ is considered the space of types as well as the space of typings.
\end{definition}

\end{document}
