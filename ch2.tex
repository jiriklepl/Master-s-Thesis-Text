
\providecommand{\cmm}[0]{C-{}- }

\chapter{C-{}-}

\todo{C-{}- is great etc etc}

Before we talk about the specifics of C-{}-, let us explain our choice of this particular language for our demonstration of using type inference and RAII in a systems-programming context.

\url{https://downloads.haskell.org\/ghc\/latest\/docs\/html\/users_guide\/codegens.html}

\xxx{man2}

Cmm (a variant of C-{}-) is used as a backend intermediate representation for GHC, the most popular haskell compiler. C-{}- is thus designed on great experiential basis by one of the co-creators of the haskell language, Simon Peyton Jones, alongside with GHC and itself.

Just as Cmm, \cmm is designed to serve as a back-end representation for various front-end compilers, abstracting away various architectural specifics, while also offering the users to target for specific registers-sizes. For example, \lstinline{bits32}; register types, in the context of \cmm, called ``kinds'' (but we will refer to them as ``data kinds'' or ``data-kinds'' to distinguish them from kinds we know from the HM type system and similar systems), note that these data kinds are usually strictly separated from each other in other languages (by so-called ``aliasing rules'' \xxx{link aliasing rules for c++ (perhaps C)? or for some asssembler if such exist?}); and then \cmm can give constraints even on using specific registers. This was one of the main motivations for choosing this language for our exploration as it strives to be minimalistic compared to high-level languages while very robust and general, compared to assemblers.

It is due to the above reasons, it is usually categorized as pseudo-assembler.

The algorithm our experimental approach is based on, so called ``deferred inference'' or the ``French approach'' \xxx{refer to the prev chapter}, was proposed to the public by the same person, so there were even more arguments for this choice motivated by some overlaps in design and ideas. \xxx{is it good to point out specific overlaps?}

As we based the proof-of-concept implementation on the second, and the latest \xxx{when?}, revision of the standard, we will always refer to this version unless otherwise specified.

\section{C-{}- specification}

What characterizes \cmm the best is perhaps simply minimalism. The syntax is almost directly a sharp subset of any other C-like language, it supports only bit-vector types with no distinction between integers and float numbers, and it offers a high number of additional compiler hints and constraints the user can specify.

What differs the approach of \cmm from other C-like languages is that it has call statements instead of call expressions. This is an important distinction as procedure calls usually perform some hidden side-effects (they have non-pure computation model), hidden in the sense that the side effects would be specified in the language syntax. This follows a similar philosophy to people like Linus Torvalds, for example, who oppose quite popular modern features like operator overloading and C\# accessors. \xxx{link torvals}, \xxx{link the cis language?}.

It is important to note that \cmm it is not minimalistic in all aspects as it, as opposed to C and other similar languages, offers so-called ``continuations''. Those are a generalization of label statements that can be called (using a ``cut to'' statement) from one procedure to jump to another. It requires a pointer to the other procedures' activation and, opposed to regular labels, it can be called with values that change the local state of the target procedure: the actual arguments of this ``cut to'' call are bound to local variables inside the activation that is to be executed.

It might seem that these continuations serve as a thread model for the language, but it is not so, they do not implement the whole thread model and the languages does not specify any thread scheduler, but it offers a well-defined hooks for the front-end compilers.

\todo{put focus on the general specifics}


\section{Used extensions to the language}

\todo{how it is extended to achieve our goals}

\subsection{Type classes and instances}

Syntactically, we extended the language with \li{class}es and their \li{instance}s to showcase the solution to the problems identified in our previous work. \xxx{explain, link}

\xxx{example for classes and motivation}


\subsection{Structures and member fields}

And then we added \li{struct}s (a record structure syntax known from other C-like languages) as they are not a part of the \cmm language and we also added \li{new} specifiers for their instances to implement our RAII goal. For these RAII features see the subsection \ref{RAII}

Structures use the syntax derived from regular data specialization for maximum consistency.

\xxx{an example for structs and motivation}

One of the side-effects of this approach is that one field can have various name aliases. This might seem inconsequential, but it can have very interesting consequences in polymorphic programming.

\xxx{more rambling?}

\xxx{one of the good features or just an interesting tid-bit of trivia?}


In our previous work, we discussed that the standard Damas-Milner approach to inference does not cover so called ``functional dependencies'', more on that in \xxx{somewhere in the first section}, there we discuss why these functional dependencies are an important topic in the scope of this thesis.

Our example implementation, albeit incomplete demonstrates the effects on programming convenience within type-safety, which comes from using functional dependencies $\text{structure} \to \text{field}$. As we discussed in the previous work, the best work-around without this inference feature would be requiring some special structure of the code: either requiring all field to have unique names; requiring all field of the same name be of the same type as well; or have the same dependencies on the structures-type arguments.

\todo{rename to records?}

\subsection{Generic dereference and typed labels}

On the semantic level, we extended the language to have typed labels. All data labels have the type derived from the object stored at the label's location, and the code labels have a distinct type from data labels as they do not point to any such object.

This typing of labels allowed us to make quite subtle, but very useful syntactic extension of allowing the type be omitted in pointer dereference expressions.

\xxx{example here}

This combines well with structures and their fields.


\xxx{example here, more text}

\subsection{Extensions to the type system}

We extended the language with the HM-like type system with type classes. And for this project, we, inspired by the optional data-kind specifications supported by the \cmm language, alongside with requirements on various values being computed in compile-time or link-time, then decided to model those in the inference as well. This was very experimental and it served as a side goal in exploring type systems and type inference.

This turned out to be a very significant decision as the modelling of these data kinds and constnesses in the context of type inference is regarded as an open question and there were not enough easily accessible guides on best practices. \xxx{we will offer such a guide in our conclusions}.

We developed a constrained subtyping mechanism extending the HM type system with type classes and type constructors, that type-checks and type-infers what types of registers a variable can be stored to and when, in terms of run-time, link-time and compile-time, the variable can be assigned a value.

This subtyping mechanism was designed to model an orthogonal space where each dimension is a bounded semilattice and we do not consider constraints with sharp inequalities. Sharp inequalities would be very easy way to design an incomputable system, and also, they do not make any sense when we want to determine just the minimal ``run-timeness'' \xxx{i cannot think of anything better to name the idea of being a run-time variable} or the minimal set of registers a value can be received on, from an assignment.

We represent the various subtype dimensions of the type variables with so-called ``subtype variables''. We will use these two representation synonymously.


\begin{defn}[Typing dimension]
    \label{typing_def}
    The typing dimension is a semilattice with instantiation \xxx{ref to instantiation} serving as the $\leq_T$ operation and $\forall a . a$ being the bottom element (we assume equivalence under renaming the bound variables \xxx{mention de bruijn notation? Are types $\forall a . \forall b . a \to b$ and $\forall b . \forall a . a \to b$ equivalent? Not really if assuming type applications, and at the same time yes they are if we are talking HM}).

    Note that, apart from other, subtype, dimensions, the equality in typing dimension is congruent with type applications, type substitutions, and, if we are not considering type functions, all type operations are bijective.
\end{defn}

\begin{defn}[Types]
    Types follow the same rules as the typing dimension, see the previous definition (\ref{typing_def}).
\end{defn}

\begin{defn}[Subtype dimension]
    Each subtype dimension $s$ is required to be a bounded semilattice orthogonal to other dimensions and non-affecting typeclass instantiation.
    
    It is defined by $\leq_s$ operation and a set of distinct subtyping constants specific to each subtype dimension. This set is nonempty for each dimension and it contains a bottom $\bot_s$.
    
    We can then simply extend the definition to apply to user-defined subtype dimensions, but that is not within the scope of this thesis.

    In the type system specific to this thesis, we use two subtype dimensions: \textit{data kinds} and \textit{constnesses}.
\end{defn}

\begin{defn}[Subtype constraints]
    We define the set of constraints $t \leq_s t'$ where $t$ and $t'$ are two types and $\leq_s$ is ordering of $t$ and $t'$ in the subtype dimension $s$
\end{defn}

\begin{defn}[Equality of types implies equality of their subtype dimensions]
    If two types $t$ and $t'$ are equal, then $t =_s t$ for every subtype dimension $s$. Note that it does not the other way.
\end{defn}

\begin{defn}[(Direct) subtype]
    We call $t$ a (direct) subtype of $t'$, if $t \leq_s t'$ in subtype dimensions and $t =_T t'$ (their typings are the same).
\end{defn}

\begin{defn}[Subtype variable]
    A subtype variable of a type variable $t$ is a variable $s$ such that $t =_s s$ and, for every other type $t' \neq t$ and a corresponding subtype variable $s'$, $t' =_s t \To s =_s s'$.
\end{defn}

Lattice-constrained subtyping is efficiently computable if it is orthogonal to the mechanism of deciding type class instances. For each topological component of the code (a part of code, that has cyclic references), it is a simple minimization problem on a semilattice structure defined by some fixed constants.

\begin{cor}
    We say that there is an inequality (or equality) constraint between two values if such constraint is given by the input to the algorithm, is in a set of base assumptions (relations between subtyping constants, definition of bottom), or comes from transitivity, reflexivity, and weak antisymmetry.

    We will always assume generated closure over transitivity, reflexivity and weak antismymmetry as that simplifies many statements and lemmata.
\end{cor}

\begin{ex}
\label{subtyping-idea}

Let us explain this on a simple procedure with arguments' constnesses $a$ and $b$ and a return value's $c$, and let there be a constant $K$ that represents a concrete constness (perhaps, it can be a link-expr requirement).

If there is a constness value $x$ and then another two constness values $y, w$ and the system contains constraints: (we will omit saying more-or-equal in order to make the statements shorter; it is good for clarity to read ``more constant'' as ``more specific'')

\begin{itemize}
    \item $x$ is more constant than $y$ (perhaps, because we assign $x$ to $y$, an easy exercise is to show that for these types of subtyping, assignment targets have to be more general than the source operand)
    \item $y$ is more constant than $c$
    \item $x$ is more constant than $K$
    \item $w$ is more constant than $x$
    \item $a$ and $b$ are more constant than $w$
\end {itemize}

\centerline{\includegraphics[width=\linewidth / 2]{img/out/subtyping-idea.pdf}}

If those are the only subtyping constraints in the given component (in this dimension, since they are required to be orthogonal), we can safely unify the constness variables $x$ and $w$ as they are limited by the same fixed values (upper-bouned by $c$ and $K$ and lower-bouned by $a$ and $b$). And we can clearly see that the minimal $x$ will be always equal to $w$ regardless of constnesses of $a$, $b$ and $c$ in various outside contexts. Less obvious is whether we can safely unify $x$ and $y$. And since the system is based on minimization, the minimal $y$ will be, again, always equal to $x$ (with an exception of cases where $x$ is not solvable; but we are interested in solving the whole constraint space and not single values, so separating them does not make sense - yet still, in practice, we can keep some outside meta-information about the bounds of the original variables for better error messages).

So, to clearly conclude the example. We are interested in an algorithm that unifies variables $x$, $y$ and $w$ and annotated the type of the function with constraints $c \geq a, c \geq b, K \geq a, K \geq b$ for the outside subtype information and $c \geq x, K \geq x, x \geq a, x \geq b$ for the inside subtype information. We then will go even further and state that $x = \sup \{a, b\}$ and that $x \leq c, x\leq K$ ($x$ is defined by its lower-bounds), notice the equality in the former constraint. We then can completely remove the constraint $x \leq c, x \leq K$ as it is redundant (can be shown via transitivity).  We then will use a notation similar to the latter constraint for more understandable outside subtype information for each variable. For $c$, that would be $c \geq \sup \{a, b\}$. Fixing the subtype variables to suprema of their lower-bounds has an effect of not having any free variables in the constraints for the procedure.

If the theory contained constraints with sharp inequalities, we could not separate the outside subtype information from the inside subtype information, nor have such liberty in unifying the variables.

This example demonstrates that the minimal subtypes within a space of fixed values can be transformed into an equality theory.

\end{ex}

We are using the term ``fixed values'', but we have not yet introduced its meaning. They are some constants that are not to be optimized by the subtyping inference mechanism. In the most simple version of this system, if we do not consider type classes (only the HM type system with type constructors), the fixed values consist of subtyping dimension constants (in the example \ref{subtyping-idea}, that would be $K$); and then the subtype dimensions of skolem constants (again, in the example \ref{subtyping-idea}, represented by the symbols $a$, $b$, and $c$).

\begin{defn}[fixed value in subtyping without type classes]
    Each subtype variable of a skolem constant or subtyping dimension constant is a fixed value.
\end{defn}

If the type is a monotype (and thus it is a valid type for a program), it has no skolem constants and that means we can safely define all the subtype variables as suprema of their lower bounds.

\begin{lemma}[Each subtype variable can be set to a suprema of its lower bounds, where each such bound is the first fixed value on each path from this variable towards the bottom; this will give us the minimimal solution to the constraints]
    \label{suprema-subtyping}
\end{lemma}

\begin{proof}
The validity of this simplification can be shown in toplogical order of the variables, where the topology is defined by their inequalities. If they are in a strongly connected component, they have to be equal, so without a loss of generality, we can assume, that there is a topological ordering of the variables, starting with the lowest.

If a set $X$ denotes the set of all the lower bounds of a variable $x$ and it is the variable with the lowest topological order. That means that if a variable $y$ constrained by $x$, then the constraint is $x \leq y$. If $x^\star$ is equal to the supremum of its lower bounds, then for every other $x'$ such that $x' \geq x^\star$, if $y \geq x'$, then also $y \geq x^\star$. This means that $x^\star$ is at least as good as $x'$ for all the constraints regarding $x$, while it also minimizes $x$ itself.
\end{proof}

\begin{defn}[Outside subtype information (of a function)]
    The outside subtype information consists of subtype constraints between all fixed values. And the bindings to their respective types. All subtype variables appearing in a strongly connected component are unified. 

    It can be reduced under transitivity and reflexivity.
\end{defn}

\begin{cor}[Outside subtype information]
    The bindings of types to their corresponding subtype variables are $1 \to \star$.
\end{cor}

\begin{defn}[Inside subtype information]
    The inside subtype information consists of a set of statements in the form $x = \sup X$.

    Validity of this is described by lemma \ref{suprema-subtyping}.
\end{defn}

\begin{cor}[Typechecker for user-code that contains blackbox function does not need inside subtype information]

    Another direct consequence of lemma \ref{suprema-subtyping}.
\end{cor}

If we extend this mechanism to the type system with typeclasses, we have to regard the subtype dimensions of variables constrained in a type class constraints.

\begin{defn}[fixed value in subtyping with type classes]
    Each subtype variable of a skolem constant, subtyping dimension constant, or a subtype variable of a type variable appearing in a type class constraint is a fixed value.
\end{defn}


The algorithm we based our type inference on uses context levels that measure how free the variables are in the given context. \xxx{see ch1}

\begin{defn}[fixed value in subtyping with type classes and context levels]
    Each subtype variable of a skolem constant, subtyping dimension constant, a subtype variable of a type variable appearing in a type class constraint, or a variable with a lower context level than that of the solver is a fixed value.
\end{defn}

We did not strive for representing existentials as there has to be runtime type information included with them, so we will ignore such system.

\xxx{the folllowing is obsolete}

As our algorithm was based on \textit{deferring inference} algorithm which assumes an equality theory of the types and not lattice structures, the implementation progressed for a long time with some modifications to the general idea of the original algorithm, but met many obstacles and there was a lot of time spent on resolving those. It then seemed necessary to offer a better, working alternative, which we will in the \xxx{reference}.

We eventually designed a simpler and much easier to implement primitives for this type system. 

\subsection{RAII}

\label{RAII}

\xxx{example for structs and motivation; why RAII, what is RAII, why is RAII, how RAII differs in the context of a procedural language}


\todo{reference something about RAII}

\section{Modelling C-{}-}

