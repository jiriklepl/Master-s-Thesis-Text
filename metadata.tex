

%%% Choose a language %%%

\newif\ifEN
\ENtrue   % uncomment this for english
%\ENfalse   % uncomment this for czech

%%% Configuration of the title page %%%

\def\ThesisTitleStyle{mff} % MFF style
%\def\ThesisTitleStyle{cuni} % uncomment for old-style with cuni.cz logo
%\def\ThesisTitleStyle{natur} % uncomment for nature faculty logo

\def\UKFaculty{Faculty of Mathematics and Physics}
%\def\UKFaculty{Faculty of Science}

\def\UKName{Charles University in Prague} % this is not used in the "mff" style

% Thesis type names, as used in several places in the title
%\def\ThesisTypeTitle{\ifEN BACHELOR THESIS \else BAKALÁŘSKÁ PRÁCE \fi}
\def\ThesisTypeTitle{\ifEN MASTER THESIS \else DIPLOMOVÁ PRÁCE \fi}
%\def\ThesisTypeTitle{\ifEN RIGOROUS THESIS \else RIGORÓZNÍ PRÁCE \fi}
%\def\ThesisTypeTitle{\ifEN DOCTORAL THESIS \else DISERTAČNÍ PRÁCE \fi}
%\def\ThesisGenitive{\ifEN bachelor \else bakalářské \fi}
\def\ThesisGenitive{\ifEN master \else diplomové \fi}
%\def\ThesisGenitive{\ifEN rigorous \else rigorózní \fi}
%\def\ThesisGenitive{\ifEN doctoral \else disertační \fi}
%\def\ThesisAccusative{\ifEN bachelor \else bakalářskou \fi}
\def\ThesisAccusative{\ifEN master \else diplomovou \fi}
%\def\ThesisAccusative{\ifEN rigorous \else rigorózní \fi}
%\def\ThesisAccusative{\ifEN doctoral \else disertační \fi}



%%% Fill in your details %%%

\def\ThesisTitle{Inference-driven resource managemenent and polymorphism in systems programming}
\def\ThesisAuthor{Jiří Klepl}
\def\YearSubmitted{2022}

% department assigned to the thesis
\def\Department{Department of Software Engineering}
% Is it a department (katedra), or an institute (ústav)?
\def\DeptType{Department}

\def\Supervisor{RNDr. Miroslav Kratochvíl, Ph.D.}
\def\SupervisorsDepartment{Department of Software Engineering}

% Study programme and specialization
\def\StudyProgramme{Computer Science - Software Systems}
\def\StudyBranch{ISWSP}

\def\Dedication{%
Dedication. \xxx{It is nice to say thanks to supervisors, friends, family, book authors and food providers.}
}

\def\AbstractEN{%
Systems programming languages facilitate the implementation of software that runs in restricted environments close to the hardware, such as operating systems, drivers, and real-time and embedded systems.
Implementation of desirable features of such languages, such as generic programming and type system capabilities that prevent programmer errors, are complicated by strict constraints on the run-time properties of the program.
This thesis explores a novel combination of C-\phantom{}- language with advanced type system features that allow type checking of highly polymorphic generic code and demonstrates type-driven resource management in this language.
As the main result, the thesis provides a proof-of-concept in a prototype compiler of an extended version of C-\phantom{}- to LLVM and describes a type system based on deferred constraint solving that is capable of type inference in the presence of multi-parameter typeclasses and C-\phantom{}- subtypes.
We demonstrate the functionality of the type system and the compiler on selected program examples and report several identified design challenges that may be addressed to make the system more practical.
}

\def\AbstractCS{%
Systémové programovací jazyky usnadňují implementaci softwaru, který běží v omezených prostředích blízko hardwaru, jako jsou operační systémy, drivery, systémy reálného času a vestavěné systémy.
Implementace prospěšných prvků pro tyto jazyky, jako jsou generické programování a typový systém zajišťující bezchybnost kódu, je ztěžována úzkými požadavky na běhovýmé vlastnostni programu.
Hlavní náplní této práce je prozkoumat novou kombinaci jazyka C-\phantom{}- s prvky typového systému umožňující typovou kontrolu vysoce polymorfního kódu a demonstruje možnosti typově vedené správy zdrojů v tomto jazyku.
Jako hlavní výstup práce podává důkaz konceptu zprostředkovaný prototypem překladače rožšířeného jazyka C-\phantom{}- do LLVM a popisuje typový systém založený na odkládajícím řešení omezujících podmínek, který je schopen typové inference za přítomnosti více-parametrových typových tříd a subtypy C-\phantom{}-.
Funkcionalitu typového systému a překladače demonstrujeme na vybraných ukázkác programů a popisujeme několik rozpoznaných komplikací návrhu, které lze adresovat pro praktičtější implementaci.
}

% 3 to 5 keywords (recommended), each enclosed in curly braces.
% Keywords are useful for indexing and searching for the theses by topic.
\def\Keywords{%
{programming languages}, {type systems}, {compilers}, {systems programming}, {resource management}
}

% If your abstracts are long and do not fit in the infopage, you can make the
% fonts a bit smaller by this setting. (Also, you should try to compress your abstract more.)
% Alternatively, consider increasing the size of the page by uncommenting the
% geometry modification in thesis.tex.
\def\InfoPageFont{}
%\def\InfoPageFont{\small}  %uncomment to decrease font size

\ifEN\relax\else
% If you are writing a czech thesis, you additionally need to fill in the
% english translation of the metadata here!
\def\ThesisTitleEN{Inference-driven resource managemenent and polymorphism in systems programming}
\def\DepartmentEN{Department of Software Engineering}
\def\DeptTypeEN{Department}
\def\SupervisorsDepartmentEN{Department of Software Engineering}
\def\StudyProgrammeEN{Computer Science - Software Systems}
\def\StudyBranchEN{ISWSP}
\def\KeywordsEN{%
{programming languages} {type systems} {compilers} {systems programming} {resource management}
}
\fi
