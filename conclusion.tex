
\chapwithtoc{Conclusion}

\section{Generic programming in \cmm{} with MPTC}

We have already demonstrated \xxx{ref to chap 1} that multi-parameter typeclasses with functional dependencies are very powerful tool And that they are instrumental for fully implemented support of member fields of record data types.

Their implementation is showcased by structs and their fields, which are a part of our extension to the base \cmm{} language.

In the conclusion, you should summarize what was achieved by the thesis. In a few paragraphs, try to answer the following:
\begin{itemize}
\item Was the problem stated in the introduction solved? (Ideally include a list of successfully achieved goals.)
\item What is the quality of the result? Is the problem solved for good and the mankind does not need to ever think about it again, or just partially improved upon? (Is the incompleteness caused by overwhelming problem complexity that would be out of thesis scope\todo{This is quite common.}, or any theoretical reasons, such as computational hardness?)
\item Does the result have any practical applications that improve upon something realistic?
\item Is there any good future development or research direction that could further improve the results of this thesis? (This is often summarized in a separate subsection called `Future work'.)
\end{itemize}
