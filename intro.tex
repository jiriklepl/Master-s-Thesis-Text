
\chapwithtoc{Introduction}

System programming languages address a wide spectrum of challenges that programmers face while developing various kinds of low-level software that typically closely cooperates with hardware, such as operating systems, drivers, real-time mission-critical systems, embedded platforms and high-performance computation software.
A common systems programming language (such as C) provides sufficient level of abstraction over the platform to avoid typical errors from repetitive programming in assembly, while providing practical guarantees about its code, typically fitting into a highly restricted run-time environment and resource limitations.

Features of such languages have to be carefully selected to reflect these limitations.
The available languages (including C, C++, Rust, Nim, Pascal, HLASM, and many others \xxx{TODO NEJAKEJ REVIEW REF}) may provide portable structured programming that frees the programmers from the necessity to code assembly directly, a type system that is able to e.g. infer types of subexpressions so that they do not need to be declared manually, a static checker (often connected to the type system) that prevents many kinds of programming errors, and facilities for generic programming that promote code reuse.

Here, we loosely follow up on our previous thesis~\cite{klepl2020type} --- we have designed and implemented a variant of a C programming language that supported generic programming driven by a polymorphic type system with typeclass-based overloading.
The system worked, but we have identified several challenges.
Mainly, the type inference for record type members and const-qualifiers would require a type system much stronger than the originally used THIH~\cite{jones1999typing}, and the base language (C) lacked support for more flexible control flow constructions.

The main goal of this thesis is to address these challenges: We extend \cmm, a low-level C-like programming language that was designed as a backend for Haskell compilers, by constructions that allow the programmers to express typeclass-based overloading with multi-parameter typeclasses. That gives the type system sufficient power to infer types of complicated language constructions, including the typed record accesses. To this language, we add a simple typeclass-driven system for automated resource management (similar to resource initialization in C++), and describe and implement an extension of the type system for managing the subtypes (`kinds') specific to \cmm.

As the main result, we have created a prototype of the compiler of the extended \cmm. The prototype is carried out in Haskell, and implements the type inference, monomorphization, several control flow analyses and a subsequent code generation to LLVM. The thesis additionaly provides a comprehensive description of the type inference algorithm based on deferred constraint solving inspired by \citet{vytiniotis2011outsidein}, extended by the subtype constructions required for \cmm. As a partial side result, our subtyping system can be extended to any, even user-specified lattice-based subtype systems, but the subtype inference must be independent on typeclass resolution to avoid complexity issues~\cite{tiuryn1992subtype,frey1997subtype} and we consequently did not pursue a practical implementation.

We show that the compiler prototype is able to process meaningful programs and output adequate LLVM code, and demonstrate the goals of the thesis on small examples. Because the development of the prototype has shown several deficiencies of the compiler and inference algorithm design that were not obvious before later stages of the project, such as the interplay of type inference with monomorphization and subtyping, we also briefly report on the drawbacks introduced by some of the design decisions, and draft several possible future improvements.

\paragraph{Thesis layout.}
The thesis is organized as follows:
\cref{chap1} reviews algorithm W and its suitability for implementation of complicated type system features, and describes the `French approach' to type inference with deferred constraint solving that allows straightforward inference of multi-parameter typeclasses and existential types.
\cref{chap2} describes \cmm and our extensions of the language for types and resource management.
\cref{chap3} details the design of the compiler and gives a comprehensive description of the type inference algorithm.
\cref{chap4} shows the main results, mainly the functionality of the compiler on selected code examples.
The outcomes and recommendations for future development are briefly summarized in \cref{chap5}, together with several highlights and `lessons learned' from the compiler development process.
We conclude with an overview of thesis goals, achievements, and future work.
